
\documentclass[12pt]{article}
\usepackage[a4paper, margin=1in]{geometry}
\usepackage{setspace}
\usepackage{titlesec}
\usepackage{lmodern}
\usepackage{enumitem}
\usepackage{hyperref}
\usepackage{graphicx}
\usepackage{times}
\usepackage{fancyhdr}
\pagestyle{fancy}
\fancyhf{}
\rhead{\thepage}
\lhead{Md Jamil Ahmed Sayem Khan}
\cfoot{}
\renewcommand{\headrulewidth}{0.4pt}

\titleformat{\section}{\bfseries\large\uppercase}{}{0em}{}
\titleformat{\subsection}{\bfseries}{}{0em}{}

\setstretch{1.5}

\title{\bfseries Research Proposals for Master's by Research\vspace{-2ex}}
\author{\normalsize Md Jamil Ahmed Sayem Khan \\ \normalsize Proposed Supervisor: Prof. Rajiv Padhye \\ \normalsize Program: Master of Technology (Fashion \& Textiles) by Research}
\date{}

\begin{document}

\maketitle
\thispagestyle{fancy}

\section*{Proposal 1: Reengineering Cigarette Filter Waste for Sustainable Protective and Medical Textiles}

\subsection*{Background \& Motivation}
Cigarette filters are among the most abundant and persistent forms of plastic pollution, composed primarily of cellulose acetate. Globally, over 4.5 trillion cigarette filters are discarded annually. Despite their biodegradable nature, they are rarely recycled. This project proposes to recover, purify, and repurpose these filters into functional materials for protective clothing and medical wearables, addressing both environmental and healthcare challenges.

\subsection*{Research Objectives}
\begin{itemize}[noitemsep]
    \item Recover and purify cellulose acetate from post-consumer cigarette filters
    \item Process purified fibers into nonwovens using electrospinning or melt blowing
    \item Functionalize resulting textiles with antimicrobial and filtration properties
    \item Test protective performance (barrier efficiency, breathability, tensile strength)
    \item Assess environmental impact through Life Cycle Analysis (LCA)
\end{itemize}

\subsection*{Methodology}
\begin{itemize}[noitemsep]
    \item Collect and sterilize discarded filters; extract cellulose acetate via solvent processing
    \item Spin fibers into mats; treat with silver nanoparticles or natural antimicrobials
    \item Characterize material using SEM, FTIR, tensile testing, and EN 14683/ISO PPE standards
    \item LCA comparison between recycled and conventional mask or gown materials
\end{itemize}

\subsection*{Expected Outcomes}
\begin{itemize}[noitemsep]
    \item A novel sustainable textile derived from waste, suitable for PPE applications
    \item A prototype PPE item (e.g., biodegradable surgical mask or medical gown)
    \item Research contribution in sustainable materials and healthcare design
\end{itemize}

\subsection*{Relevance to RMIT and CAMPT}
This research directly supports RMIT CAMPT's mission in circular design, technical textile innovation, and sustainable protective clothing. With expertise in polymer-based performance textiles and medical applications, Prof. Padhye is an ideal mentor.

\newpage

\section*{Proposal 2: Heat-Regulating Smart Fabric for Firefighter PPE Using PCM-Infused Yarns}

\subsection*{Background \& Motivation}
Firefighters operate in extreme heat environments with limited thermal comfort. Traditional PPE suits often lack adaptive thermal regulation, contributing to fatigue, heat stress, and injury. This research proposes to integrate Phase Change Materials (PCMs) into multilayered textile systems to passively manage heat loads. PCMs can absorb, store, and release thermal energy during phase transitions, improving wearability and protection.

\subsection*{Research Objectives}
\begin{itemize}[noitemsep]
    \item Develop yarns or coatings containing microencapsulated PCMs (e.g., paraffin waxes)
    \item Integrate PCMs into firefighter PPE fabric systems (liners or outer layers)
    \item Test thermal performance, flame resistance, and mechanical durability
    \item Model temperature distribution and heat flux under simulated fire conditions
\end{itemize}

\subsection*{Methodology}
\begin{itemize}[noitemsep]
    \item Select appropriate PCM with melting point \textasciitilde 30--35\textdegree C
    \item Use microencapsulation techniques and apply to yarns via coating or extrusion
    \item Laminate or knit into protective multilayer textiles
    \item Conduct thermal analysis (TGA, DSC), flammability tests (ISO 15025), and comfort tests
\end{itemize}

\subsection*{Expected Outcomes}
\begin{itemize}[noitemsep]
    \item Development of smart PPE fabric that actively manages body heat
    \item Reduction in core body temperature in simulation tests
    \item High-impact publication in advanced materials or PPE safety research
\end{itemize}

\subsection*{Relevance to RMIT and CAMPT}
RMIT's CAMPT has a strong track record in PPE innovation and textile performance systems. This project aligns with Prof. Padhye's ongoing research in firefighter apparel, smart textiles, and material science for safety applications.

\end{document}
